\documentclass{article}
\author{
    Alex Manning
    \and
    Ryan Meuth
    \and
    Kevin Burger
}
\title{Analysis of Circuit-Board-Level Radiation-Hardening Techniques for a Prompt-Dose Environment}

\begin{document}
\maketitle
\tableofcontents
\pagebreak

\section{Topic Overview}\label{Topic Overview}
    The prompt-dose radiation environment provides a unique set of challenges for electronic design that can not always be solved using conventional rad-hardening (radiation hardening)
    methods.  Prompt-dose refers to a radiation environment in which a high dose of radiation is delivered to a device over a very short period of time.  A pragmatic approach to rad-hardening a
    device against prompt-dose radiation would be of particular interest to devices used in  defense applications with a requirement to continue functionality in the event of a nuclear detonation.  The prompt-dose environment 
    threatens the functionality of electronics primarily due to gamma rays, high-energy neutron radiation, and the electromagnetic pulse (EMP) created due to reactions in Earth's upper atmosphere.
    These effects can cause component degradation, especially in semiconductors, as well as transients that can temporarily or permanently damage circuit operation if left unhandled.\\

    A pragmatic approach to prompt-dose rad-hard design should provide future designers a set of practical tools to design a circuit that is resistant to prompt-dose radiation effects.
    This should include the following as necessary:
    \begin{itemize}
        \item An anlysis of the possible effects of prompt-dose radiation exposure on common electrical components based on previously published literature
        \item Guidelines and recommendations for selecting appropriate components
        \item Guidelines and recommendations for appropriate schematic design
        \item Guidelines and recommendations for appropriate circuit-board layout
        \item Guidelines and recommendations for Field-Programmable-Gate-Array (FPGA) design
    \end{itemize}

\section{Research Plan}
    Research will begin with a thorough literature review of prompt-dose radiation effects on electronics based on existing literature.  Using the knowledge gained, a specification will be generated for a DC-DC converter
    with at least the following features:
    \begin{itemize}
        \item Support for a wide range of input voltages
        \item Multiple switchable output voltages
        \item Appropriate onboard command and monitoring circuitry
        \item An appropriate interface connect to an offboard monitor for circuit verification during test
    \end{itemize}
    The board will be manufactured, tested, analyzed, and iterated as resources permit.  The project will culminate in a report as described in section \ref{Topic Overview}.

\section{Meetings}
    Meetings will occur upon request of any party as required to ensure successful completion of this project.  A status update will be delivered at least every two weeks in the form of an email to
    all committee members.  Participation in Ryan Meuth's Honors Thesis Organizational Canvas Course will be completed as required.

\section{Timeline}
    \begin{tabular}{|c|c|}
        \hline
        \textbf{Date} & \textbf{Due} \\\hline
        09/27/2020 & Literature Review \\\hline
        10/04/2020 & Board Specification \\\hline
        11/01/2020 & Board Design/Fabrication Files $\cdot$ Test Procedure \\\hline
        11/15/2020 & Completed Board Units $\cdot$ Radiation Test Scheduled \\\hline
        11/22/2020 & Board Functional Test Results \\\hline
        12/13/2020 & Board Radiation Test Results \\\hline
        01/10/2021 & Test Result Analysis \\\hline
        01/17/2021 & Updated Board Specification \\\hline
        01/31/2021 & Updated Board Design/Fabrication Files $\cdot$ Updated Test Procedure  \\\hline
        02/14/2021 & Completed Updated Board Units $\cdot$ Radiation Test Scheduled \\\hline
        02/28/2021 & Board Functional Test Results \\\hline
        03/14/2021 & Board Radiation Test Results \\\hline
        03/21/2021 & Report Completed \\\hline
        04/02/2021 & Defense Completed \\\hline
        04/16/2021 & Thesis Submitted \\\hline
    \end{tabular}

\end{document}